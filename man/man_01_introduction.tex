\chapter{Introduction}

Candy Makefile is a is a platform a makefile ready to build C/C++ application, \LaTeX documents,
and many other things to come. The idea is to code all building pipeline once, and then use it
every where by only few lines.


\section{Setup}

\subsection{Step 1: Clone from Github}

The first thing you have to do is, well... Download lastest Candy Makefile by simply cloning
its official repository: \url{http://github.com/gabadie/makefile}

Using Git:
\begin{lstlisting}[language=bash]
> git clone https://github.com/gabadie/makefile.git candy_makefile
\end{lstlisting}

Or using Subversion:
\begin{lstlisting}[language=bash]
> svn checkout https://github.com/gabadie/makefile candy_makefile
\end{lstlisting}

Notes: You may be intesrted by cloning its repository in the project's repository your are begining
to work on.
\begin{lstlisting}[language=bash]
> git submodule init
> git submodule add https://github.com/gabadie/makefile.git candy_makefile
\end{lstlisting}


\subsection{Step 2: Create your makefile}

Now you need to create your makefile, but that is going to be very easy because it will only
include a Candy Makefile's .mk file:

\begin{lstlisting}[language=make]
#
# include Candy Makefile's entry point
#
include candy_makefile/build.mk
\end{lstlisting}


\subsection{Step 3: Create the project configuration file}

The project configuration file is where all appen, where you are going to create all your project's
products. Only one thing to note: It has to be in the current working directory, besides of the
makefile we have just created right before.

\begin{lstlisting}[language=bash]
> touch candy_project.mk
\end{lstlisting}


\subsection{Step 4: Edit the project configuration file}

Now you will just need to edit your project configuration file as need for your products. For example,
let us say that you want only a PDF product generated from a \LaTeX source code for a MAN. Then, you
will create just the following lines:

\lstinputlisting[language=make]{candy_project.mk}

And your \LaTeX document is now ready to be built with:

\begin{lstlisting}[language=bash]
> make
\end{lstlisting}

And then, just open it by~;

\begin{lstlisting}[language=bash]
> make open/all
\end{lstlisting}

