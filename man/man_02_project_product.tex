\chapter{Project's products}

Candy Makefile let you to have severals products in your same product. Each product can
be of differents types, but each product can only have one target at a time. The product's
target is the what your are going to have at the very end of the differents pipelines, and
they can be changes, depending on the configuration.

Candy Makefile brings a object oriented API to manipulate very easily all differents
project's products. This entire chapter is dedicated to this API.


\section{Create a product}

Hey, wait a minute... I'm supposed to create the project, not Candy Makefile!
What does 'create a product' mean? Easy: Like all object oriented API, before manipulating
an object, you need to create (or instantiate) it. Therefore creating a product is only
creating the virtual object associated to your real product.

For that purpose, you have the following function returning a new product:

\begin{lstlisting}[language=make]
$(call product_create,<product_type>,<product_name>)
\end{lstlisting}

\begin{description}
    \item[Parameter product\_type] is the type of the product. For example, if you want a
    product generated from a \LaTeX source code, then it should be \textit{LATEXPDF}.
    \item[Parameter product\_name] is the name of the product your are creating. It must be
    unic for each product. Example: \textit{foo\_bar}.
    \item[Returns] the newly created product object.
\end{description}

Example:
\begin{lstlisting}[language=make]
FOO_BAR_PRODUCT := $(call product_create,LATEXPDF,foo_bar)
\end{lstlisting}


\section{Make a product public}

A public product is a product that you can build directly with the command lines
\textit{build/update} or \textit{build/full}. Indeed, you may not want to build testing
products when you want to building product that you are actually going to release.

\begin{lstlisting}[language=make]
$(call product_public,<product_obj>)
\end{lstlisting}

\begin{description}
    \item[Parameter product\_obj] a valid product object.
    \item[Returns] nothing.
\end{description}

Example:
\begin{lstlisting}[language=make]
$(call product_create,$(FOO_BAR_PRODUCT))
\end{lstlisting}


\section{Get the product's target}

The product's target is what your are actually going to build in the product directory.
It is what you want to configure.

\begin{lstlisting}[language=make]
$(call product_target,<product_obj>)
\end{lstlisting}

\begin{description}
    \item[Parameter product\_obj] a valid product object.
    \item[Returns] the product's target.
\end{description}

Example:
\begin{lstlisting}[language=make]
FOO_BAR_TARGET := $(call product_target,$(FOO_BAR_PRODUCT))
\end{lstlisting}

