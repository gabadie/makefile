\chapter{Commands}

\section{Build commands}

\subsection{Update all public product}

\begin{lstlisting}[language=bash]
> make
> make build/update
\end{lstlisting}

\begin{description}
    \item[Action:] Updates all public products' targets if needed.
\end{description}


\subsection{Update a specific product}

\begin{lstlisting}[language=bash]
> make build/product/<product_name>
\end{lstlisting}

\begin{description}
    \item[Action:] Updates \textit{product\_name}'s target if needed.
    \item[Parameter product\_name:] An existing product's name.
\end{description}


\subsection{Forces build from scratch}

\begin{lstlisting}[language=bash]
> make build/full
\end{lstlisting}

\begin{description}
    \item[Action:] Forces compilation of all public product's target from scratch.
\end{description}


\subsection{Recursively force build from scratch}

\begin{lstlisting}[language=bash]
> make build/full/rec
\end{lstlisting}

\begin{description}
    \item[Action:] Recursively forces compilation of all public product's target from scratch.
\end{description}


\section{Mr proper commands}

\subsection{Clean build directory}

\begin{lstlisting}[language=bash]
> make clean
\end{lstlisting}

\begin{description}
    \item[Action:] Cleans the entire build directory, except for logs' and products' directories.
\end{description}


\subsection{Delete build directory}

\begin{lstlisting}[language=bash]
> make trash
\end{lstlisting}

\begin{description}
    \item[Action:] Deletes the entire build directory and its content.
\end{description}


\subsection{Clobber build}

\begin{lstlisting}[language=bash]
> make clobber
\end{lstlisting}

\begin{description}
    \item[Action:] Deletes the entire build directory and its content, and update the clobber file.
\end{description}


\section{Open commands}

\subsection{Open a product}

\begin{lstlisting}[language=bash]
> make open/product/<product_name>
\end{lstlisting}

\begin{description}
    \item[Action:] Open a valide openable product.
    \item[Parameter product\_name:] An existing product's name.
\end{description}


\subsection{Open all openable products}

\begin{lstlisting}[language=bash]
> make open/all
\end{lstlisting}

\begin{description}
    \item[Action:] Open all openable products.
\end{description}


\section{Run commands}

\subsection{Run a product}

\begin{lstlisting}[language=bash]
> make run/product/<product_name>
\end{lstlisting}

\begin{description}
    \item[Action:] Run a valide executable product.
    \item[Parameter product\_name:] An existing product's name.
\end{description}


\subsection{Run all openable products}

\begin{lstlisting}[language=bash]
> make run/all
\end{lstlisting}

\begin{description}
    \item[Action:] Run all executable products.
\end{description}

